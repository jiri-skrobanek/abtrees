\documentclass[a4paper,11pt,openany]{article}
\usepackage[utf8]{inputenc}
\usepackage[czech]{babel}
\usepackage{geometry}
\usepackage{tikz,multirow,listingsutf8,multicol,amsmath,amsfonts}
\lstset{showstringspaces=false,stringstyle=\color{orange},keywordstyle=\bfseries\color{blue!60!black},}
\geometry{left=20mm,right=20mm,top=20mm,bottom=20mm}
\parindent=0mm
\parskip=0mm
\usepackage{lipsum}
\newcommand{\quotefont}[2]{
#1 [online]. [Citováno \today].\\
Dostupné z:\\\small \ttfamily #2 \normalfont
}
\newcommand{\float}{\texttt{float}}

\begin{document}
\begin{center}
	\pagenumbering{gobble}
	{\huge \textbf{(a,b)-stromy}}\\\vspace{\baselineskip}Zápočtová práce z Programování I pro pokročilé\\
	\vspace{10mm} {\large Jiří Škrobánek\footnote[1]{Matematicko-fyzikální fakulta Univerzity Karlovy, {\ttfamily jiri@skrobanek.cz}}}\\
	\vspace{10mm}\today, Ostrava
\end{center}

\section*{Abstract}

This documentation describes the entire functionality of (a,b)-trees implementation in Python 3 by Jiří Škrobánek. Aside from listing all methods, principles of (a,b)-trees are explained and complexity of used algorithms is analysed.

\tableofcontents

\section{Definice}
(a.b)-strom je strom. Musí platit $(a,b) \in \mathbb{N}^2, 2 \leq a, 2a - 1 \leq b $. (a,b)-strom je buďto prázdný, nebo mají všechny vnitřní vrcholy nejméně $a$ synů a nejvýše $b$ synů. Výjimkou je kořen, jenž musí mít mezi 2 a $b$ syny. Všechny listy leží v jedné hladiny. Vnějším vrcholům jsou přiřazeny unikátní klíče (prvky lineárně uspořádané množiny). Vnitřním vrcholům je přiřazen maximální klíč z jeho synů. Ve vnitřním vrcholu jsou uloženy klíče synů ve vzestupném pořadí. Pro každý podstrom platí, že mimo podstrom neexistují vnější vrcholy, které mají nižší klíč než maximální klíč v podstromu a zároveň vyšší klíč než minimální v~podstromu.

V tomto stromě se dá vyhledat vnější vrchol dle klíče v logaritmickém čase vzhledem k počtu vnějších vrcholů. Přidávání a odebírání listů rovněž funguje v logaritmickém čase.

Speciálním případem (a,b)-stromů $a$ jsou

\section{Metody}
\begin{lstlisting}[language=python,frame=none]
__init__(a: int,b: int)
\end{lstlisting}
\begin{lstlisting}[language=python,frame=none]
insert(key: int, value=None)
\end{lstlisting}
\section{Popis algoritmů}

\lipsum[10]

\lipsum[10]

\section{Analýza složitosti}

\lipsum[10]

\lipsum[10]

\listoffigures
\listoftables
\section*{Seznam příloh}
%\addcontentsline{toc}{chapter}{Seznam příloh}
\begin{enumerate}
	\item[A.] Zdrojový kód knihovny
	\item[B.] Zdrojové kódy příkladů
\end{enumerate}
\begin{thebibliography}{10}
	\bibitem{aocp}
	KNUTH, Donald Ervin. The Art of Computer Programming. Upper Saddle River, NJ: Addison-Wesley, 2011. ISBN 978-0321751041.
\end{thebibliography}
\end{document}